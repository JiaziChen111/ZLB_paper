\documentclass[12pt,reqno]{article}
\usepackage[utf8x]{inputenc}
\usepackage{graphicx}
\usepackage[table]{xcolor}
\graphicspath{ {figures/} }
\usepackage[margin=0.9in]{geometry}
\usepackage{amsmath}
\usepackage{amssymb}
\usepackage{subfigure}
\usepackage[english]{babel}
\usepackage[T1]{fontenc}
\usepackage{amsmath,amsfonts,amssymb}
\usepackage{pdflscape}
\usepackage{natbib}
\usepackage{verbatim}
\usepackage{lipsum}
\usepackage{gensymb}
\usepackage{comment}
\usepackage{bm}
\usepackage{changepage}
%\usepackage{amsart}
\usepackage{lscape}
\usepackage{soul}
\usepackage{float}
\usepackage{rotating}
\usepackage{graphicx}
\usepackage{setspace}
\usepackage[color]{changebar}
\linespread{.5}
\numberwithin{equation}{section}
\usepackage{appendix}
\usepackage{authblk}
\usepackage{hyperref}
\usepackage[most]{tcolorbox}



\hypersetup{colorlinks,allcolors=blue}

\title{Restricted Perceptions and the Zero Lower Bound Episode \footnote{The research project is funded by the National Bank of Belgium under \textit{the Research Programme for Young Researchers}. The views expressed in this paper are those of the author and do not necessarily reflect the views 
of the National Bank of Belgium or any other institution to which the author is affiliated.}\vspace{10 mm} \\ (Preliminary and incomplete, do not cite or distribute) }
\author{Tolga Ozden\footnote{T.ozden@uva.nl, University of Amsterdam}, Rafael Wouters\footnote{Rafael.wouters@nbb.be, National Bank of Belgium}}
\begin{document}

\maketitle


\bibliographystyle{apalike2}
\begin{abstract}
%\subsection{Extended Abstract}

We consider the estimation of Markov-switching DSGE models under adaptive learning to study the interaction between expectations and the business cycle over the Zero Lower Bound period. We assume that agents never directly observe regime shifts by monetary policy, instead they indirectly infer about the regimes to the extent that it feed back into their information set.  We show that in these cases, there is a so-called Restricted Perceptions Equilibrium (RPE) consistent with a given information set, and standard E-stability conditions are applicable to such equilibria. We then use a variant of the  filter to estimate MS-DSGE models under constant gain adaptive learning. Based on our estimations of the benchmark 3-equation NKPC and workhorse  models, our results are summarized as follows: adaptive learning models outperform the REE benchmark in all cases, and the Regime-switching REE model in most cases, suggesting that Markov-switching and Adaptive Learning approaches can be complementary. Furthermore, we observe important differences in the impulse response and shock propagation structure of the models under consideration. Particularly, a financial shock and a government spending shock of the same size typically has a longer-lasting impact under adaptive learning, suggesting that the Rational Expectations models may severely underestimate both the impact of 2007-08 financial crisis, as well as the impact of a fiscal stimulus during the zero lower bound episode that followed the crisis. 


\end{abstract}
